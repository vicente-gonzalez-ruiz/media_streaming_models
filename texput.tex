\title{Media streaming models}

\maketitle
\tableofcontents

\section{Chunked streams}
\begin{itemize}
\item In order to reduce the transport layer data overhead, the
  streams are packetized, and each packet contains a chunk of data.
\item The length can be constant or variable, and depends on the MTU,
  the maximum header overhead that is allowed, the buffers size at the
  receivers, etc.
\end{itemize}

\section{Push vs pull models}
\begin{itemize}
\item In push models, the sender sends the data without any feedback
  from the receiver(s).
\item In pull models, the receiver (usually called client) controls
  the data-flow, i.e., which data must send the sender (usually called
  server), depending on factors such as the available bandwidth, the
  avaiable resources at the client, etc.
\end{itemize}

\section{On demand vs live streaming}
\begin{itemize}
\item Movies are streamed on demand (VoD (Video on Demand) streaming),
  usually using pull protocols.
\item Live events are live streamed (live streaming), usually using
  push protocols.
\end{itemize}

%\section{Unicast vs multicast vs peer-to-peer transmissions}
%\begin{itemize}
%\item In unicast, the stream travels from a single sender to a single
%  receiver. 
%\item In a multicast transmission, the sender (never a server) sends a
%  copy of the stream and the network replicates the stream as many
%  times as receivers exist.
%\item In the P2P model, the sender sends a copy or a reduced number of
%  copies of the stream, and the receivers (called peers) retransmit
%  the stream as many times as necessary to allow the reception of the
%  stream to all the peers.
%\end{itemize}

%\section{Single Layer Coding (SLC) + Client/Server (C/S) Model}
%{{{

%\svg{SLC+CS}{400}

%\begin{itemize}
%\item Notice that a buffer underflow (i.e. a lost of a GOP during the
%  playback) can occur if $t_T<t_E$ (the transmission bit-rate is smaller
%  than the encoding bit-rate).
%\end{itemize}

%}}}

%\section{Media simulcast + Client/Server (C/S) Model}
%{{{

%\svg{simulcast+CS}{400}

%\begin{itemize}
%\item If the client's buffer is going to underflow, the client should
%  retrieve a stream with a smaller bit-rate (this happens for the
%  GOP$_2$).
%\end{itemize}
% \item In a solution to provide service scalability, a media server can
%   store (and serve) collection of non-scalable streams, depending on
%   the characteristics of the transmission channel and the receiver
%   (resolution, computing power, battery, etc.). Notice that
%   simulcasting produces a replication of information at the server
%   side, but not on the network nor the clients.
% \item This is used, for example, at YouTube.
% \item Currently, most video servers provide spatial scalability
%   (YouTube, for example) by means of media simulcast. However, due to
%   stream switching is not supported by the Web browsers along the
%   playing, the selection of the resolution must be done only at the
%   beginning of the transmission.
% \item Moreover, temporal and quality scalabilities are not supported
%   by current Web browsers because only single-layer (non-scalable)
%   decoders are implemented in them, even when the GOPs (that could
%   change the picture-rate or the quality) can be decoded
%   independently.
% \item However, notice that this is only a lack of functionality of the
%   majority of the browsers. This could be solved soon.

%}}}

%\section{Multiple Layer Coding (MLC) + Client/Server (C/S) Model}
%{{{

%\svg{MLC+CS}{400}

%\begin{itemize}
%\item If the client's buffer is going to underflow, the client should
%  retrieve less layers.
%\end{itemize}

%}}}

%\section{Multiple Description Coding (MDC) + Client/Server (C/S) Model}
%{{{

%\fig{400}{MDC+CS.png}

%\begin{itemize}
%\item If the client's buffer is going to underflow, the client should
%  retrieve less descriptions (this happens for the GOP$_2$).
%\end{itemize}

%}}}

%\section{Single Layer Coding (SLC) + Peer-to-Peer (P2P) Model}
%{{{

%\begin{itemize}
%\item Due to all peers need to share the same single-layered stream,
%  $t_E$ should never be bigger than $t_T$, for every peer in the cluster,
%  or a buffer underflow will occur in those peers where $t_E>t_T$.
%\end{itemize}

%}}}

%\section{Media simulcast + Peer-to-Peer (P2P) Model}
%{{{

%\begin{itemize}
%\item Each media can be simultaneously broadcasted but in different
%  channels (clusters of peers).
%\item Peers can switch between clusters depending on the transmission
%  bit-rate (channel switching should be fast in order to use
%  efficiently the available bandwidth).
%\item The time to perform a switch between channels depends on the
%  buffering time.
%, which depends on the buffer size, the transmission
%  bit-rate and the GOP-rate, that depends on the GOP size and the
%  picture-rate.
%\item In a switch, the end of the reception of the old channel should
%  coincide with the beginning of the reception of the new channel,
%  i.e, the buffering time should be predictable. Otherwise, the
%  reception of both channels must be overlapped.
% \item Due to all peers need to share the same simulcasted stream, only
%   a high bit-rate GOP should be delivered by the source if all peers
%   can receive such amount of data.
%\end{itemize}

%}}}

%\section{Multiple Layer Coding (MLC) + Peer-to-Peer (P2P) Model}
%{{{

%\begin{itemize}
%\item If each layer is transmitted over a different cluster, peers can
%  join/left to more/less clusters depending on the transmission
%  bit-rate.
%\item The transmission of the layer $X-1$ must be prioritized to the
%  transmission of the layer $X$.
% \item Because the $X$-th quality, resolution or picture-rate implies
%   the reception of all layers up to the the $X$-th one, all peers can
%   share the same scalable stream independently of the number of layers
%   received.
%\end{itemize}

%}}}

%\section{Multiple Description Coding (MDC) + Peer-to-Peer (P2P) Model}
%{{{

%\begin{itemize}
%\item As in MLC, if each description is transmitted over a different
%  cluster, peers can join/left to more/less clusters depending on the
%  transmission bit-rate.
%\item However, in this case it is not necessary to prioritize the
%  transmission of the descriptions (although a peer could be rejected
%  from a cluster (description) if it becomes unsupportive).
%\end{itemize}

%}}}

